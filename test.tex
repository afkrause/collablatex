% vorlage: https://poolmgr.informatik.uni-freiburg.de/kurs/LaTeX-Vorlagen/report-vorlage-medium.html
\documentclass[a4paper,12pt,ngerman]{report}

% Sprache, Umlaute und Silbentrennung für deutsche Dokumente
\usepackage{babel}
\usepackage[utf8]{inputenc}
\usepackage[T1]{fontenc}

% Serifenfreie Schriftart für den Fließtext
\usepackage{helvet}
\renewcommand{\familydefault}{\sfdefault}

% 1.5-facher Zeilenabstand
\usepackage{setspace}
\onehalfspacing

% Seitengeometrie für DIN A4 mit Rändern
\usepackage[a4paper, margin=2.5cm]{geometry}

% Farbige Textelemente und Hervorhebungen
\usepackage{xcolor}

% Grafiken und Bilder
\usepackage{graphicx}

% Mathematikpakete für Formeln und Symbole
\usepackage{amsmath, amssymb}

% Paket für Literaturverweise
\usepackage[numbers]{natbib}

% Kopf- und Fußzeilengestaltung
\usepackage{fancyhdr}

% Hyperlinks und URLs
\usepackage{hyperref}
\hypersetup{
    colorlinks=true,
    linkcolor=blue,
    urlcolor=blue,
    citecolor=blue,
    pdftitle={Beispielbericht in LaTeX},
    pdfauthor={Max Mustermann},
    pdfsubject={Ein Beispielbericht mit der LaTeX-Dokumentenklasse report},
    pdfkeywords={LaTeX, report, Vorlage, Beispielbericht}
}

% Intelligente Verweise
\usepackage{cleveref}

\title{Ein ausführlicher Beispielbericht in LaTeX}
\author{Max Mustermann}
\date{\today}

\begin{document}

% Titelseite
\begin{titlepage}
    \centering
    \vspace*{1cm}
    
    {\Huge \textbf{Ein ausführlicher Beispielbericht in LaTeX}}
    
    \vspace{1.5cm}
    
    {\Large Max Mustermann}
    
    \vfill
    
    Ein Bericht zur Demonstration der `report`-Dokumentenklasse in LaTeX.
    
    \vfill
    
    {\large \today}
    
\end{titlepage}

\pagenumbering{roman}
\tableofcontents
\listoffigures
\listoftables
\newpage

% Kopf- und Fußzeilen
\pagestyle{fancy}
\fancyhf{}
\fancyhead[L]{Beispielbericht}
\fancyhead[R]{\leftmark}
\fancyfoot[C]{\thepage}

\pagenumbering{arabic}

\chapter{Einleitung}
Dies ist ein Test-Satz. 
Dies ist ein Test-Satz.
Hoffentlich klappt das!
Leider müssen wir lokal arbeiten. 
Später testen ob editieren in einem chapter gemeinsam auch klappt. 

Diese Vorlage zeigt den Aufbau eines Berichts in der LaTeX-Dokumentenklasse 
`report`. Der Stil `report` ist besonders geeignet für umfangreiche Dokumente, 
da er eine Kapitelstruktur unterstützt. Die Vorlage nutzt verschiedene 
LaTeX-Pakete, um das Layout zu verbessern und hilfreiche Funktionen für 
wissenschaftliche Arbeiten bereitzustellen.

\chapter{Grundlagen}
Hier ist ein neuer Test Satz ... 

In diesem Kapitel werden grundlegende Strukturen und nützliche LaTeX-Befehle 
vorgestellt, die in formellen Dokumenten häufig benötigt werden.

\section{Mathematische Ausdrücke}
Die Pakete `amsmath` und `amssymb` bieten eine Vielzahl von mathematischen 
Symbolen und Formeln. Hier einige Beispiele für mathematische Ausdrücke:

\begin{equation}
    E = mc^2
\end{equation}

\[
\int_{a}^{b} f(x) \, dx = F(b) - F(a)
\]

\subsection{Ein Beispiel für eine Formel mit Summenzeichen}
\[
\sum_{i=1}^{n} i = \frac{n(n+1)}{2}
\]

\section{Tabellen und Abbildungen}
Tabellen und Abbildungen sind wichtige Elemente in wissenschaftlichen Arbeiten,
 um Daten und Ergebnisse anschaulich darzustellen.

\subsection{Tabellen}
Hier ist ein Beispiel für eine einfache Tabelle:

\begin{table}[!ht]
    \centering
    \begin{tabular}{|c|c|c|}
        \hline
        A & B & C \\
        \hline
        1 & 2 & 3 \\
        4 & 5 & 6 \\
        \hline
    \end{tabular}
    \caption{Beispieltabelle}
    \label{tab:beispiel}
\end{table}

\subsection{Abbildungen}
Mit dem Paket `graphicx` können Bilder einfach eingefügt werden. Das folgende 
Beispiel zeigt, wie man eine Abbildung mit Bildunterschrift und Label erstellt:

\begin{figure}[!ht]
    \centering
    \includegraphics[width=0.5\textwidth]{example-image} 
    % Ersetze "example-image" mit einem tatsächlichen Dateinamen
    \caption{Eine Beispielabbildung}
    \label{fig:beispiel}
\end{figure}

\section{Verweise und Hyperlinks}
Dank `cleveref` können Verweise auf nummerierte Elemente intelligent gesetzt 
werden. Ein Verweis auf die Tabelle \cref{tab:beispiel} und die 
Abbildung \cref{fig:beispiel} erfolgt so automatisch korrekt.

Für Hyperlinks und URLs, die durch `hyperref` und `url` unterstützt werden, 
können Sie beispielsweise \href{https://www.latex-project.org/}{LaTeX Projektseite}
 aufrufen oder eine URL ohne Verlinkung darstellen: \url{https://www.latex-project.org/}.

\section{Listen und Aufzählungen}
Listen sind nützlich, um Punkte in einer übersichtlichen Form darzustellen. 
Die häufig verwendeten Listentypen sind:

\begin{itemize}
    \item Ein Punkt
    \item Noch ein Punkt
\end{itemize}

\begin{enumerate}
    \item Schritt 1
    \item Schritt 2
\end{enumerate}

\begin{description}
    \item[Begriff 1:] Eine kurze Erklärung des Begriffs.
    \item[Begriff 2:] Weitere Erklärung.
\end{description}

\chapter{Schlussfolgerungen}
Die Vorlage demonstriert, wie man ein formales Dokument in LaTeX aufbauen kann. 
Die Kombination der genutzten Pakete ermöglicht eine flexible und professionelle
 Gestaltung, die für umfangreiche Berichte und wissenschaftliche Arbeiten ideal ist.

\newpage

% Literaturverzeichnis
\begin{thebibliography}{9}
    \bibitem{latexcompanion} 
    Michel Goossens, Frank Mittelbach, and Alexander Samarin. 
    \textit{The \LaTeX\ Companion}. 
    Addison-Wesley, 1993.
    
    \bibitem{einstein} 
    Albert Einstein.
    \textit{Zur Elektrodynamik bewegter Körper}. (German) 
    [\textit{On the electrodynamics of moving bodies}]. 
    Annalen der Physik, 322(10):891–921, 1905.
    
    \bibitem{knuthwebsite} 
    Knuth: Computers and Typesetting,
    \url{http://www-cs-faculty.stanford.edu/~uno/abcde.html}
\end{thebibliography}

\appendix
\chapter{Anhang}
Im Anhang können zusätzliche Materialien, wie lange Tabellen, detaillierte 
Berechnungen oder weitere Diagramme, eingefügt werden. Der Anhang ist besonders 
hilfreich, um den Hauptteil der Arbeit übersichtlich zu halten.

\section{Erweiterte mathematische Darstellung}
Hier ist ein Beispiel für eine mehrzeilige Gleichung:
\begin{align*}
    a + b &= c \\
    a - b &= d
\end{align*}

\end{document}
